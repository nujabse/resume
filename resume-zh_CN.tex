% !TEX TS-program = xelatex
% !TEX encoding = UTF-8 Unicode
% !Mode:: "TeX:UTF-8"

\documentclass{resume}
\usepackage{zh_CN-Adobefonts_external} % Simplified Chinese Support using external fonts (./fonts/zh_CN-Adobe/)
% \usepackage{NotoSansSC_external}
% \usepackage{NotoSerifCJKsc_external}
% \usepackage{zh_CN-Adobefonts_internal} % Simplified Chinese Support using system fonts
\usepackage{linespacing_fix} % disable extra space before next section
\usepackage{cite}


\begin{document}
\pagenumbering{gobble} % suppress displaying page number

\name{朱蒙}

\basicInfo{
  \email{h.p.zhumeng@gmail.com} \textperiodcentered\ 
  \phone{(+86) 156-2910-0557} \textperiodcentered\ 
  \linkedin[mengzhu]{https://www.linkedin.com/in/meng-zhu-40b368b6/}}
 
\section{\faGraduationCap\  教育背景}
\datedsubsection{\textbf{华中科技大学}, 物理学院}{2017 -- 至今}
\textit{在读硕士研究生}\ 凝聚态物理, 预计 2020 年 6 月毕业
\datedsubsection{\textbf{武汉工程大学}, 材料科学与工程学院}{2012 -- 2016}
\textit{学士学位}\ 材料物理

\section{\faUsers\ 实习/项目经历}

\datedsubsection{\textbf{Lab data}}{2018 年 9 月 -- 至今}
\role{Python}{个人项目}
\begin{onehalfspacing}
  使用 numpy 和 scipy 对实验数据进行批量分析 
  \begin{itemize}
    \item 快速绘制 I-T、I-V 曲线
    \item 自动标记分析力和电流信号的峰值谷值
  \end{itemize}
\end{onehalfspacing}

\datedsubsection{\textbf{Pymonitor}}{2019年4月 -- 至今}
\role{Python}{个人项目}
\begin{onehalfspacing}
使用 PyQt5 开发的实验仪器使用情况登记助手, https://github.com/nujabse/PyMonitor
\begin{itemize}
  \item 确保使用实验仪器前必须登记
  \item 对实验仪器使用时段进行可视化分析
\end{itemize}
\end{onehalfspacing}

\datedsubsection{\textbf{Journal App}}{2019 年5月 -- 至今}
\role{Flutter}{个人项目}
\begin{onehalfspacing}
每日心情记录 App,使用跨平台的 Flutter 框架开发, https://github.com/nujabse/Journal-App
\begin{itemize}
  \item 跨 Android 和 iOS 双平台
\end{itemize}
\end{onehalfspacing}

\section{\faCogs\ IT 技能}
% increase linespacing [parsep=0.5ex]
\begin{itemize}[parsep=0.5ex]
  \item 编程语言: Python, Java
  \item 平台: Linux
\end{itemize}

\section{\faHeartO\ 获奖情况}
\datedline{湖北省翻译大赛 \qquad \textit{非专业组二等奖}}{2013 年 11 月}
\datedline{华中科技大学 \quad\qquad \textit{研究生一等硕士学业奖}}{2017--2018 学年}
\datedline{华中科技大学 \quad\qquad \textit{三好研究生}}{2017--2018 学年}
\datedline{华中科技大学 \quad\qquad \textit{优秀共青团员}}{2019年 5月}

\section{\faInfo\ 其他}
% increase linespacing [parsep=0.5ex]
\begin{itemize}[parsep=0.5ex]
  \item \textbf{语言}: 英语 - 熟练(\textbf{TOEFL: 91,  CET4: 641, CET6: 571}), 日语 - 入门
  \item \textbf{软件技能}: 熟练使用 Origin 数据处理软件,熟练使用 Photoshop,Illustrator, CorelDRAW 等图像处理软件以及 3ds MAX, Cinema 4D 等三维建模渲染软件
  \item \textbf{爱好}: 折腾 Archlinux 和 Android 手机刷机, 自行车骑行, 以及希区柯克的电影
  \item \textbf{技术博客}: https://nujabse.github.io
  \item \textbf{Github}: https://github.com/nujabse
  \item \textbf{软件技能}: 熟练使用 Origin 数据处理软件,熟练使用 Photoshop,Illustrator, CorelDRAW 等图像处理软件以及 3ds MAX, Cinema 4D 等三维建模渲染软件
\end{itemize}

%% Reference
%\newpage
%\bibliographystyle{IEEETran}
%\bibliography{mycite}
\end{document}
